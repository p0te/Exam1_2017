\documentclass{article}
\title{REII414 past papers comaprison}
\author{MJ Bezuidenhout}
\begin{document}
\maketitle
\section{2015}
\begin{itemize}
	\item 6 questions:
		\subitem 1:sql Theory
		\subitem 2:entity and referential integrity
		\subitem 3: ERD diagram
		\subitem 4: Dependecies, noprmailzation
		\subitem 5: difficult qrys (joins, sums etc)
		\subitem 6: essay: final year project
\end{itemize}
\section{2016}
\begin{itemize}
	\item 5 questions:
		\subitem 1: sql Theory 
		\subitem 2: entity and referential integrity
		\subitem 3: Dependencies, normalization
		\subitem 4: Difficult queries,(joins, sums etc)
		\subitem 5: essay: game
\end{itemize}
\section{Sql Theory}
\begin{itemize}
	\item Data vs information
	\item Right vs left joins
	\item ERD principles(Arrow types, redundant relationship)
	\item Derived attribute
	\item attributes of a PK
	\item triggers
	\item SQL injection
\end{itemize}
\section{Entity and referential integrity}
\begin{description}
	\item[Data Integrity] In a relational database, refers to a condition
in which the data in the database is in compliance with all entity and referential integrity constraints 
	\item[Entity Integrity]  The property of a relational table that guarantees that each entity has a unique value in a
primary key and that there are no null values in
the primary key.
	\item[Referential Integrity] A condition by which a dependent tables foreign key must have either a null entry or a matching entry in the related table
\end{description}
\section{Dependencies, Normalization}

\section{Difficult queries}

\section{Essay question}
\begin{itemize}
	\item Database Design
		\subitem ERD
	\item Web programming
		\subitem UI
		\subitem data-interfacing		
		\subitem security
\end{itemize}
\end{document}
